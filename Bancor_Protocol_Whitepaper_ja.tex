\documentclass{jsarticle}

\usepackage[dvipdfmx]{graphicx,color,hyperref}
\usepackage{amsmath}

\begin{document}

\title{
  \textbf{Bancorプロトコル}
  \protect\linebreak
  \protect\linebreak
  \large
  スマートコントラクトを通じて、トークンの継続的な流動性を確保し、非同期的な価格発見を可能にするスマートトークンについて
}

\author{Eyal Hertzog, Guy Benartzi \& Galia Benartzi}
\date{May 30, 2017}

\maketitle

\begin{center}
  \item[訳者:] \href{https://kentarok.org}{栗林 健太郎}
  \item[原本:] \href{http://www.hyuki.com/girl/}{Draft Version 0.99}
\end{center}

\begin{figure}[b]
  「欲求の二重一致問題」は、Jevons (1875)によって提起された。

  \begin{quotation}  
    取引は二者間で、かつ、その二者において処分可能な所有物がお互いの欲求を満たす場合に可能となる。欲求を抱く多くの人々が存在し、そして、欲求されるべき多くの所有物が存在する。しかし、現に取引が行われるためには、まれにしか起こらない、欲求の二重一致が必要になる。
  \end{quotation}
\end{figure}

% 目次
\newpage
\tableofcontents
\newpage

\section{Bancorプロトコル}

\emph{
  要約: Bancor(バンコール)プロトコルには、スマートコントラクト上のトークンのための、価格発見
  \footnote{
    $\href{
      https://en.wikipedia.org/wiki/Price_discovery
    }{
      https://en.wikipedia.org/wiki/Price\_discovery
    }$
  }
  と流動性を担保するメカニズムが備わっている。スマートトークンはひとつ以上のトークンを準備金として保有し、その準備トークンと交換することで、誰もが即座にスマートトークンを購入したり、精算したりすることができる。 そしてそれは、スマートコントラクトを直接的に通じて行われ、継続的に計出される価格において取引される。その計算は、売買をバランスする公式に基づいて行われる。
} \\

Bancorプロトコルは、第二次世界大戦後、国際的な通貨の換算をシステム化するための、Bancorと呼ばれる超国家的な準備通貨の導入に関する、ケインズ経済学者(たち)の提案
\footnote{
  $\href{
    https://en.wikipedia.org/wiki/Bancor
  }{
    https://en.wikipedia.org/wiki/Bancor
  }$
}
に敬意を表して名付けられた。

  \subsection{背景}

  我々は、誰もが記事・歌あるいは動画を公開し、ディスカッションのためのグループを形成し、さらにはオンラインのマーケットプレイスを運営できるような、そんな世界に住んでいる。いまや我々は、一般の人々によって生成された通貨の発生をも目の当たりにし始めている。様々な形で貯蔵された価値(以下、通貨と呼ぶ)が、何世紀にもわたって発行され、流通してきた。それらは、紙幣、債権、株式、ギフトカード、ポイント、コミュニティ通貨など、様々な形態をなしてきた。ビットコインは、初の\emph{非中央集権的な}デジタル通貨であり、その後には暗号通貨発行の流行が続いた。また近頃では、資産の新しい形としての「トークン」の隆盛があり、それらは典型的にはクラウドセール(ICO)において、スマートコントラクトを通じて発行されている。

  しかしながら、通貨とは本質的には\href{https://blog.bancor.network/coins-are-networks-and-crowdsales-are-their-killer-app-a6ebc16bef31}{ネットワークの価値}を示すものであり、情報のネットワークにおいて行われるようには、お互いにつながったりはしないものである。情報のネットワークにおいては、インターネット上の情報の交換地点であるスイッチが情報を連結する一方で、通貨においては、取引所にいる活発なトレーダーが通貨を連結する。
  
  現在の通貨あるいは資産の交換モデルには、致命的な障壁がある。それは、市場の流動性を得るためにある程度の量の取引が必要になるということである。この生来の障壁は、コミュニティ通貨
  \footnote{
    $\href{
      https://en.wikipedia.org/wiki/Community_currency
    }{
      https://en.wikipedia.org/wiki/Community\_currency
    }$
  } 
  やポイント、あるいはその他の特別仕様のトークンのような小規模な通貨が、市場において決定される交換レートに基づいて、その他の人気のある通貨と交換されることをほとんど不可能にする。

  スマートコントラクトブロックチェーンの時代においては、発行やふるまいを制御する不変のコードによって、トークンは自動的に管理される。このことは、トークンの製作者によって設計され、自動的に管理されるスマートコントラクトを直接的に通じて、トークンが他のトークンの残高を保有する(すなわち、備えておく)ことができるということである。これらの新しい技術的な可能性は、あり得べき通貨交換のソリューションや市場価格の決定について再考する、十分な理由となる。

  \subsection{スマートトークンの導入:流動性問題の解決策}

  スマートトークンは、標準的なERC20トークンであり、Bancorプロトコルを実装することで、継続的な流動性と自動的かつ円滑な価格発見を両立する。スマートトークンのコントラクトは、即座に\emph{売り注文}と\emph{買い注文}とを処理し、そのことで価格発見のプロセスが駆動される。この能力により、スマートトークンは流動性の確保において、取引所での取引を必要としない。

  スマートトークンは、少なくともひとつ、それ自身以外のトークンを\emph{準備トークン}として保有する。それは、(現在のところ)別のスマートトークンや、ERC20に準拠したトークン、あるいはEtherでもあってもよい。スマートトークンは、購入された際に発行され、流動した際に破棄される。したがって、それはいつでも準備トークンによってその時点の価格で購入され得るし、同様に、準備トークンへと精算され得る。

  \subsection{価格発見の新手法}



  \subsection{スマートトークンのユースケース}



    \subsubsection{ユーザにより作成される通貨のロングテール}



    \subsubsection{プロジェクトのクラウドファンディング}



    \subsubsection{トークン交換所}



    \subsubsection{非中央集権的なトークンバスケット}



    \subsubsection{ネットワークトークン}



  \subsection{スマートトークンの利点}



  \subsection{Bancorプロトコルのエコシステム}



  \subsection{「欲求の二重一致問題」への解決策}



    \subsubsection{スマートトークンの初期化とカスタマイゼーション}



\section{Bprotocolファウンデーション}



  \subsection{Bancor Network Token (BNT):初めてのスマートトークン}



  \subsection{BNTをクラウドセールスする目的}



\section{トランザクションごとの価格計算}



\section{要約}



\section{謝辞}



\end{document}
