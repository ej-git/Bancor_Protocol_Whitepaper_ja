\documentclass{jsarticle}

\usepackage[dvipdfmx]{graphicx,color,hyperref}
\usepackage{amsmath}

\begin{document}

\title{
  Bancor Protocol(バンコール・プロトコル) \\
  \large
  スマートコントラクトを通じて、トークンの継続的な流動性を確保し、非同期的な価格発見を可能にするスマートトークンについて
}

\author{Eyal Hertzog, Guy Benartzi \& Galia Benartzi}
\date{May 30, 2017}

\maketitle

\begin{center}
  \item[訳者:] 栗林 健太郎
  \item[原本:] \href{http://www.hyuki.com/girl/}{Draft Version 0.99}
\end{center}

\begin{figure}[b]
  「二重の一致問題」は、Jevons (1875)によって提起された。

  \begin{quotation}  
    取引は二者間で、かつ、その二者において処分可能な所有物がお互いの要求を満たす場合に可能となる。要求を抱く多くの人々が存在し、そして、要求されるべき多くの所有物が存在する。しかし、現に取引が行われるためには、まれにしか起こらないような、二重の一致が必要になる。
  \end{quotation}
\end{figure}

% 目次
\newpage
\tableofcontents
\newpage

\section{Bancor Protocol(バンコール・プロトコル)}



\subsection{スマート・トークン入門:流動性問題の解決策}



\subsection{価格発見の新手法}



\subsection{スマート・トークンのユースケース}



\subsubsection{ユーザにより作成される通貨のロングテール}



\subsubsection{プロジェクトのクラウドファンディング}



\subsubsection{トークンの交換所}



\subsubsection{トークン交換所}



\subsubsection{非中央集権的なトークン・バスケット}



\subsubsection{ネットワーク・トークン}



\subsection{スマート・トークンの利点}



\subsection{バンコール・プロトコルのエコシステム}



\subsection{「二重の一致問題」への解決策}



\subsubsection{スマート・トークンの初期化とカスタマイゼーション}



\section{Bprotocolファウンデーション}



\subsection{Bancor Network Token (BNT):初めてのスマート・トークン}



\subsection{BNTをクラウドセールスする目的}



\section{トランザクションごとの価格計算}



\section{要約}



\section{謝辞}




\end{document}
